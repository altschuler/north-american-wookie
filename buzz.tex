%%% Local Variables:
%%% mode: latex
%%% TeX-master: "cheat-sheet"
%%% End:

\subsection*{Week 2 - C Programs}
\begin{itemize}
\item A Java program is a set of Classes
\item What does C programs consists of?
\item NB: There are major differences compared to Java
\end{itemize}

\subsection*{Week 3 - C Programs part 2}

Memory - Buzz

\begin{itemize}
\item How is the memory in a computer typically organized?
  \begin{enumerate}
    \item How is it addressed?
    \item How does it look like to the processor?
    \item What memory areas do a program use?
    \item How are they laid out? In what order? Where in memory?
    \item Single threaded

  \end{enumerate}
\end{itemize}

Memory I - Buzz

Where do
\begin{itemize}
\item Global variables go?
\item How is it addressed?
\item How does it look like to the processor?
\end{itemize}

Pointers - Buzz

\begin{itemize}
\item Fundamentally, what is a pointer?
\item How is it represented?
\item How many bits are used?
\item Is type associated with pointers?
\item How do you do type casting?
\end{itemize}

Answers:

\begin{itemize}
\item  Fundamentally, pointers are represented as the smallest integer which can hold an address
– Describes an address and how to interpretate it
\item Represented using the (*) symbol. E.g. {\tt char *string; }
\item Like this: {\tt <++>ppi = (double ∗)pn; /∗ pn originally of type ( int ∗) ∗/ }
\end{itemize}



