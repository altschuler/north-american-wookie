%%% Local Variables:
%%% mode: latex
%%% TeX-master: "cheat-sheet"
%%% End:

Situations where two or more processes are reading or writing some shared data and the final result depends on who runs precisely when, are called \emph{race conditions}. Remember, processors are completely \emph{independent}.

A \emph{critical race} occurs when the order in which internal variables are changed determines the eventual state that the state machine will end up in.

A \emph{non-critical race} occurs when the order in which internal variables are changed does not alter the eventual state.

Race conditions creates \emph{Indeterministic} behavior, where the resulting state can not be determined.

Recreating a bug caused by a race condition might take some extreme set of parameters and are therefore very hard to find.

If shared variables, operations need to be \emph{synchronized}.

\subsection{Example}
\begin{align*}
\texttt{int x = 0;} \\ \\
&\textbf{P}_1 &\textbf{P}_2 \\
&\texttt{x = x + 1} &\texttt{x = 3}
\end{align*}

Due to the various possible interleavings, the final value of \texttt{x} can be 1, 3 or 4.

\subsection{Example with atomic statements}
\begin{align*}
\texttt{int x = 0;} \\ \\
&\textbf{P}_1 &\textbf{P}_2 \\
&\big \langle \texttt{x = x + 1} \big \rangle &\texttt{x = 3}
\end{align*}

As the two processes can only be executed sequentially now, the possible final values of \texttt{x} are 3 or 4.
