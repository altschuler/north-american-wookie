%%% Local Variables:
%%% mode: latex
%%% TeX-master: "../cheat-sheet"
%%% End:

\subsubsection*{Intro}
\begin{itemize}
\item What does a program use during its lifetime?
  \begin{itemize}
  \item Hint: what does a program need to execute?
  \end{itemize}
\end{itemize}

\paragraph{Answer}
See previous answer.

\subsubsection*{System calls}

\emph{What is a system call?}
\paragraph{Answer}
\begin{itemize}
\item Programming interface to the sercices proided by the OS.
  \begin{itemize}
  \item Processes $\rightarrow$ the operating system
  \end{itemize}
\item Mostly accessed by programs via a higlh-level Application Program Interface (API) rather than direct system call use.
\item Three most common APIs are Win32 API for windows, POISX API for POISX API for POISX-Based systems (including virtually all versions of UNIX, Linux, and Mac OS X), and Java API for the Java virtual machine (JVM).
\end{itemize}

\emph{How Are system calls implemented?}
\begin{itemize}
\item What are the steps in executing a system call?
\item How are parameters to system calls passed to the operating system?
\item Does the type and organization of the operating system matter?
\item Draw a figure to help you.
\end{itemize}

\paragraph{Answer}
Execution
\begin{enumerate}
  \item Set registers to system call arguments and ID (\texttt{EAX})
  \item Execute interrupt
  \item Enter kernel-mode
  \item Look-up system call and execute
  \item TODO: continue patty
\end{enumerate}

\begin{itemize}
  \item Typically, a number associated with each system call.
  \item The system call interface invokes intended system call in OS kernel and returns status of the system call and any return values.
  The caller need know nothing about how the systemm call is implemented.
  \begin{itemize}
    \item Just needs to obay API and understand what the OS will do
  \end{itemize}
\end{itemize}
