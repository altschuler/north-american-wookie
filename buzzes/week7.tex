%%% Local Variables:
%%% mode: latex
%%% TeX-master: "cheat-sheet"
%%% End:

\subsubsection*{Forms of scheduling}
One form of scheduling is to map threads, and hence processes, to CPUs.
(What Simon was trying to express here, is to map process \# 1 to CPU \# 1, process \# 2 to CPU \# 2, etc.)

What are other forms?



\subsubsection*{How do you do a context switch?}
Store registers, including stack pointer and instruction pointer, so that the execution of the thread can be continued later on.

\subsubsection*{Bonus: Fragmentation}
\begin{description}
\item[External Fragmentation]: total memory space exists to satisfy a request, but it is not contiguous (in other words: it is fragmented)
\item[Internal Fragmentation]: allocated memory block may be slightly larger than requested memory
\end{description}
Fragmentation \texttt{will} happen