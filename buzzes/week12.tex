%%% Local Variables:
%%% mode: latex
%%% TeX-master: "../cheat-sheet"
%%% End:

\subsection*{Week 12 - Virtual Machines}
\subsubsection*{Why virtual machines and virtual machine monitors?}

\begin{itemize}
	\item You have plenty of installations...
	\begin{itemize}
		\item Servers
		\item Operating systems versions
	\end{itemize}

	\item ... but not much hardware
	\begin{itemize}
		\item Hardware == Energy == Money
		\item Many installations are idle
	\end{itemize}
	\item Usage scenario: Web hosting
	\begin{itemize}
		\item Fewer machines to manage
		\item Hopefully not less reliable or less secure
	\end{itemize}
	\item Hypervisor can control a number of guest operating systems
\end{itemize}

\subsubsection*{Migration}
``One advantage of using virtual machines is that you can move a virtual machine from one hardware machine to another. How could one do that?''

Image that you have two machines; \emph{Machine A} and \emph{Machine B}. The user using is running \emph{Machine A}, but the overall system decides that the virtual machine should be moved to \emph{Machine B} (probably due to \emph{Machine A} is failing).

The job is to make the transition as smooth as possible for the user. This is achieved by copying all of the memory from \emph{Machine A} to \emph{Machine B}. But this is not done in a random manner:
\begin{enumerate}
	\item Firstly, the memory of the processes \emph{not} used by the user is copied from \emph{Machine A} to \emph{Machine B}
	\item When all but the critical memory has been copied (after a some seconds), the process will shortly freeze as the last data it is moved from \emph{Machine A} to \emph{Machine B}
	\item Now the user's entire virtual machine is running on \emph{Machine B}
\end{enumerate}
