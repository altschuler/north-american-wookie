%%% Local Variables:
%%% mode: latex
%%% TeX-master: "../cheat-sheet"
%%% End:

% 9. You can show with assembly code how the stack pointer, or a general purpose register, can be used to perform indirect addressing of data structures.

Consider
\begin{lstlisting}[language=C]
struct person {
  double height;
  int    age;
}

struct person turing = (struct person*)malloc(sizeof(struct person));
turing->age = 40;
turing->name = "Alan Turing";
\end{lstlisting}

We can now use indirect addressing to access the fields of the struct:

\begin{lstlisting}[language={[x86masm]Assembler}]
; eax now holds the address of `turing'
mov turing, %eax

; mov 1.0 to the first field of the struct (the double literal is not valid assembly)
mov $1.0, (%eax)

; mov 2 to the second field which is offset by 8 bytes (size of the double)
mov $2, 8(%eax)
\end{lstlisting}


Generally:
\begin{description}
\item[Indirect addressing:] \texttt{(\%eax)}
\item[Base pointer addressing:] \texttt{4(\%eax)}
\end{description}

The general form of memory address references is this:

\texttt{ADDRESS\_OR\_OFFSET(\%BASE\_OR\_OFFSET,\%INDEX,MULTIPLIER)}

\texttt{FINAL ADDRESS = ADDRESS\_OR\_OFFSET + \%BASE\_OR\_OFFSET + MULTIPLIER * \%INDEX}

See Bartlett chapter 3
