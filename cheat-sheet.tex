%%% Local Variables:
%%% mode: latex
%%% TeX-master: t
%%% End:

\documentclass{article}

\usepackage{fullpage}
\usepackage[utf8]{inputenc}
\usepackage{listings}
\usepackage{caption}
\usepackage{subcaption}
\usepackage[svgnames]{xcolor}
\usepackage{amssymb}
\usepackage{amsmath}
\usepackage{fancyhdr}
\usepackage{lastpage}
\usepackage{parskip}
\usepackage{abstract}
\usepackage{gensymb}
\usepackage{url}
\usepackage{float}
\usepackage{enumitem}
\usepackage{amstext}
\usepackage{fancybox}
\usepackage{amsmath}
\usepackage{graphicx}
%\usepackage{subfigure}
\usepackage[bottom]{footmisc}
\usepackage{hyperref}
\usepackage{tikz}
\usepackage{makecell}
\usepackage{tabulary}
\usepackage{pdfpages}
\usepackage{verbatim}
\usepackage{tikz}
\usetikzlibrary{positioning}
\usetikzlibrary{arrows}
\pagenumbering{gobble}

\setcounter{secnumdepth}{3} % only chapter and sections will be numbered
\setcounter{tocdepth}{3}    % entries down to \subsubsections in the TOC

\title{Operating System Notes}
\author{December 2014}
\date{}

\begin{document}

\maketitle
% \tableofcontents
% \pagebreak

\section*{1 - Compiler optimization}
\begin{enumerate}
	\item Loop-optimization: Move loop-invariant variables outside the loop.
	\item Dead code removal.
	\item Subexpression elimination: (x + y) / 3.0 - (x + y)
	\item Replacing instructions with more efficient ones. For instance, x = x * 2 is likely faster performed by bitshifting
\end{enumerate}

Primarily optimizing either for (run) time or for space.



\section*{3 - Explain components in compilation toolchain}
\begin{enumerate}
	\item Source file(s) are handled by the preprocessor, which turn them into \textbf{translation units}:
	\begin{itemize}
		\item The preprocesser recursively includes files marked by \#include
		\item The preprocesser expands macros such as \#if, \#define, \#ifdef
	\end{itemize}
	Translation units consists of declarations and definitions.
	\item Each translation unit is then \emph{compiled} into object files
	\item Object files can then, together with (static) libraries, be linked into executables.
\end{enumerate}

\textbf{Remember} that there are both \emph{static} and \emph{dynamic} libraries. Static are linked into the executable, whereas dynamic can be changed (updated) with time (dynamic linked).


\section*{4 - Memory leaks in systems with dynamic memory management}
\subsection*{How they can occur}
\begin{itemize}
	\item In the C standard library, one can call \texttt{malloc()} to get a pointer to the memory of the requested size.
 If one deletes the pointers to the allocated data, without freeing the memory block before, this is a memory leak.

 	\item Memory fragmentation
\end{itemize}


\subsection*{2 strategies to prevent them}
\begin{itemize}
	\item Garbage collection. Detect data objects that cannot be used by the program in the future, and free the resources occupied by these.
	\item TODO:
\end{itemize}


\section*{5 - Explain the diff. between communicating with an I/O device using memory mapped I/O or I/O-instructions}
TODO: !!!

\section*{6 - Explain processor, register, program counter, cache, I/O device, instruction set architecture}

\begin{center}
\includegraphics[width=8.0cm]{images/330px-Von_Neumann_Architecture.png}

Von Neumann Architecture
\end{center}


\subsection*{Processor}
There are different kinds of processors, the most widely used are
\begin{itemize}
	\item General purpose central processing units (CPUs, used in PCs)
	\item ASIC (application-specific integrated circuit)
	\item GPU (graphics processing units)
\end{itemize}

Operates in \textbf{cycles}:
\begin{itemize}
	\item Fetch instruction (using the PC/IP)
	\item Decode instruction (there are different \emph{addressing modes})
	\item Execute - Control Unit and ALU (Arithmetic/Logic Unit)
\end{itemize}
(This is the so-called three-stage pipeline. There is also the modern superscalar CPU (see page 21 in DJ Tanenbaum's book))

\textbf{Word size} - Modern PCs use a word-size of 64 bits. Embedded systems using microcontrollers often have a smaller word size (modern ones have down to 8 bit)

\subsection*{Register}
Lulz

\subsection*{Program counter / instruction pointer (PC/IP)}
So... yeah.

\subsection*{Cache}
TODO: cache coherence - when there are multiple caches for multiple processors. These must be kept consistent.

\subsection*{I/O device}

\subsection*{Instruction set architecture (ISA)}
Well-defined software/hardware interface.

\begin{itemize}
	\item Functional definition - Operations and storage locations supported by hardware. With Intel's Ivy Bridge chip-architecture came the operation RdRand for instance, which is an operation that returns a hardware-generated random number.
	\item Documentation of how to use the instructions (see ABI below?).
\end{itemize}

(TODO: application binary interface (ABI). I believe this is contained within ISA. This defines the calling conventions for the architecture - for instance how system calls are carried out)

\section*{8 - Explain how interrupts are handled in the processor}
When the processor gets interrupted, it

\begin{enumerate}
	\item Halts execution of the current thread
	\item Stores the current state (TODO: the book says that this ranges from storing only the IP-register, to storing \emph{all} registers of the thread).
	\item Executes interrupt handler
	\item Resumes thread execution
\end{enumerate}

Two types of interrupts:
\begin{itemize}
	\item \textbf{Hardware} - when a disk-read, keyboard input, clock (timing), scanned document is ready. Sent via the Bus.
	An Interrupt Controller handles this and issues interrupt for CPU.
	(if there are several HW-interrupts sent simultaneously, the ones with lower priorities keep sending the signal)

	\item \textbf{Software} - Either when an `interrupt instruction' is executed, or when an exception is thrown. (exceptions can be from other processors as well!)

\end{itemize}

The halting of execution and storing of state is much more complex on modern, superscalar CPUs (see page 21 in DJ Tanenbaum's)


\section*{10 - draw a figure showing how the memory hierarchy of a modern computer with the processor(s), cache levels, memory and storage}

\begin{center}
\includegraphics[width=5.0cm]{images/cpu_cache_structure.png}
\includegraphics[width=5.0cm]{images/CPU-Cache-System.png}



Simplified overviews of CPU, registers, caches and memory
\end{center}

\emph{L1} is often local for each CPU, whereas \emph{L2} and/or \emph{L3} are/is often shared between multiple CPUs.

\emph{Extremely} important: When you draw the storage, remember to draw it as a cylinder\footnote{due to historical reasons}!

By avoiding cache misses (thus increasing cache hits) we can speed up memory lookups. When the CPU fetches memory it will fetch nearby elements into the cache as well because they are typically used together. If we fetch memory from scattered places its unfriendly for the cache.

For instance, iterating a two-dimensional array:

\begin{lstlisting}
nums = [[1, 2],
        [3, 4]]
// consider that `nums' this stored in memory as [1, 2, 3, 4]
\end{lstlisting}

If we iterate column-wise, and the cache fetches a single element ahead we get something like

\begin{lstlisting}
for col = 0 to 1:
    for row = 0 to 1:
        print nums[row][col]
\end{lstlisting}

This will result in accesses in this order: \texttt{nums[0][0] (1), nums[1][0] (3), nums[0][1] (2), nums[1][1] (4)}. If the CPU fetches and caches one address ahead then the first memory access fetches \texttt{nums[0][0] (1)} and caches \texttt{nums[0][1] (2)} but the next fetch is actually \texttt{nums[0][0] (3)} which results in a cache miss.

We can get two cache hits and two memory lookups instead of four memory lookups if we iterate row-wise instead!

%%% Local Variables:
%%% mode: latex
%%% TeX-master: "cheat-sheet"
%%% End:


\section*{16 - Explain how multiple processes can share the same CPU}
This is possible by context switching between the execution of the processes or threads (this is called \textbf{multitasking}). The context switch stores the current state of the process, so that the execution of it can be resumed at a later point in time.




This is most likely implemented with a preemptive scheduler on PCs, as programs would otherwise be able to hold on to the CPU. So the scheduler decides when to switch between processes.

\begin{center}
\includegraphics[width=5.0cm]{images/220px-Multithreaded_process.png}



A process with two threads - run on a single CPU
\end{center}
The above can also be performed between several processes, and thus a single CPU can execute several processes ``at the same time'' (at least it might appear so to the user).

\section*{17 - Explain how multiple processes can share a set of CPUs}
With multiple CPUs one can, instead of switching out a running process to let another run, simply run the other process on another CPU.

Given the individual process address space and individual call stack for each thread, multiple processes are able to run in parallel on several CPUs without affecting the states of one another.

\section*{Keyword - Context switching}
Stores the state of the registers, as the rest will remain untouched by other processes (
TODO: mention individual 1) process address space and 2) call stack for each thread)


\section*{C - interrupts!}
TODO: fuck fuck fuck


\section*{C - storage classes (week 2)}
TODO: auto, extern, static and register!


\section*{Memory layout (week3)}
TODO: (memory areas) How are they laid out? In what order Where in memory?



\section*{20}
Given reference literature, you can write down in text and in your own words definitions or explanations of the following concepts:

\begin{description}
\item[Process Address space]
  The actual address space taken allocated to a process in a virtual address space.

\item[Interprocess communication]
  Can be done in numerous ways: files, message queues, semaphore, message passing, shared memory, pipes (IO), and more.

\item[System call]
  A program's request to the operating system to do an OS tasks

\item[Daemon]
  A process running in the background (``invisible''), can be a watchdog-like program, auto-updater, anti-virus, etc.

\item[Thread]
  Smallest ``unit'' known by the scheduler. Owned by a process. Threads of the same process share memory.

\item[Critical section]
  A piece of code that needs mutual exclusion. A critical \texttt{region} consists of multiple of these.

\item[Mutual exclusion]
  Ya kno dis

\item[Semaphore]
  P = request/take coconut, V = put coconut

\item[Mutex]
  Binary semaphore, a simple lock/unlock mechanism

\item[Monitor]
  A construct (class, object, whatever) with mutual exclusion of methods

\item[Condition variable]
  In a Java monitor it's the \texttt{wait} and \texttt{notify} and \texttt{notifyAll}. It's the mechanism by which a monitor can wait for a condition to be true without blocking the monitor.

\item[Message passing]
  Communication between objects in the same process or between different processes by use of so-called messages. Can be asynchronous or synchronous. Invoking a method on an object in Java is an example of message passing, and sending a signal from one process to another in unix with \texttt{kill} is another.

\item[Kernel mode]
  One of the two modes of operation of the CPU. In kernel mode all code executed is \textit{trusted} so it can access all memory, signal all processes, execute all instructions, etc.

\item[User mode]
  One of the two modes of operation of the CPU. In user mode code is not assumed to be \textit{trusted} so it is restricted to a certain memory space and possibly a limited subset of CPU instructions.

\item[Pre-emtiveness]
  A preemption happens when a process is suspended \textit{during} its execution, resulting in a \textit{context switch}, usually done by the scheduler. The intent is that the process will be resumed later.

\item[Race condition]
  Happens when two processes attempts to access/mutate a shared resources at the same time, such that unexpected behavior or corruption can occur.

\item[Scheduling algorithm]
  Algorithm that chooses which processes (or threads) to execute.

\item[System time]
  The computer's time ``counter''

\item[Device driver]
  Program that controls a hardware device and provides an interface to it (usually through the bus)

\end{description}


























\end{document}