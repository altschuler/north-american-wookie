% 13. You can explain the concept of privilege levels and show three examples of operations which should be permitted only at the most privileged level.
Processes run under privilege level from 0 (most privileged) through 3 (least privileged). The assigned level controls what resources the process has access to, such as memory, IO, special instructions. Kernel code runs in ring 0 and general user programs run in level 3, all levels are not always used (for instance 1 and 2 are unused in Linux). Level 0 is essentially kernel mode. Current privilege level is stored in the \texttt{CPL} register.

Operations only permitted in level 0 (many of these are just system calls)
\begin{itemize}
\item Process creation
\item Process termination
\item Disable interrupts
\item Modify memory map (virtual memory)
\item File writing
\item File reading
\end{itemize}

%%% Local Variables:
%%% mode: latex
%%% TeX-master: "cheat-sheet"
%%% End:
